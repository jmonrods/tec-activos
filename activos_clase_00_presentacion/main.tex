\documentclass[10pt,t,aspectratio=169]{beamer}
%\usetheme{Berkeley}
\usepackage{graphicx}
\usepackage{amsmath}
\usepackage[american]{circuitikz}

\title{Elementos Activos}
\subtitle{Presentación del curso}
\author{Dr.-Ing. Juan José Montero Rodríguez}
\subject{Elementos Activos}
\institute{Escuela de Ingeniería Electrónica}
\date{Semestre II-2023}
\titlegraphic{\includegraphics[height=12mm]{figures/logoTEC.pdf}}



\begin{document}


\section{Presentación del curso}

\begin{frame}[t]
\titlepage
\end{frame}

\begin{frame}[t]
\frametitle{Información General}%
\begin{table}[H]
	\flushleft
	\begin{tabular}{ll}
	\textbf{Curso} & Elementos Activos \\
	\textbf{Código} & EL-2207 \\
	\textbf{Tipo de curso} & Teórico \\
	\textbf{Electivo} & No \\
	\textbf{Créditos} & 4 \\
	\textbf{Horas por semana} & 4 horas por semana \\
	\textbf{Horas extraclase} & 8 horas por semana \\
	\textbf{Áreas curriculares} & 100\% Ciencias de la Ingeniería \\
	\textbf{Ubicación en plan de estudios} & IV Semestre \\
	\textbf{Requisitos} & EL-2113 Circuitos Eléctricos en CC \\
	\textbf{Correquisitos} & EL-2114 Circuitos Eléctricos en CA \\
	\textbf{El curso es requisito de} & EL-3212 Circuitos Discretos \\
	\textbf{Suficiencia} & Sí \\
	\textbf{Metodología} & Clases magistrales \\
	\textbf{Asistencia} & Virtual \\
	\textbf{Vigencia del programa} & Semestre I-2024
	\end{tabular}
\end{table}
\end{frame}

\begin{frame}[t]
\frametitle{Objetivos}
\textbf{Descripción del curso}

\begin{itemize}
	\item Este curso cubre la teoría básica de los semiconductores y los dispositivos activos semiconductores más importantes, a saber, la unión PN, diodos, los transistores MOSFET y bipolares y sus aplicaciones analógicas y digitales.
	\item El estudiante logrará un conocimiento de la teoría básica de dispositivos con semiconductores, sus curvas características, modelos matemáticos, análisis y diseño de circuitos en que son empleados, como base para los cursos de Circuitos Discretos y Diseño Lógico.
\end{itemize}

\vspace{3mm}
\textbf{Objetivo general}

\begin{itemize}
	\item Explicar a nivel electrónico utilizando fundamentos de la física del semiconductor el funcionamiento los siguientes dispositivos semiconductores diodos, transistores MOSFET y bipolar, e interpretar correctamente el funcionamiento de un dispositivo (diodo, transistor, etc) a partir de sus curvas características.
\end{itemize}


\end{frame}


\begin{frame}[t]
\frametitle{Contenidos}

\textbf{Semiconductores (2.5 semanas)}

\begin{itemize}
	\item Conceptos básicos: niveles de energía, cristal, bandas de conducción, valencia, nivel de Fermi, ecuación estadística de Fermi-Dirac.
	\item Clasificación de los materiales de acuerdo con la conducción eléctrica: semiconductores, aislantes y conductores
	\item Semiconductores intrínsecos y extrínsecos, dopado, el concepto de hueco, corriente de huecos, generación y recombinación
	\item Transporte de portadores de carga: movilidad, conductividad, corriente de difusión, corriente de arrastre, relación de Einstein
	\item Modelo de bandas de energía: nivel de Fermi, afinidad electrónica, función de trabajo, nivel de vacío, concentración de portadores de carga en función de la energía, deformación de bandas
\end{itemize}

\vspace{3mm}
\textbf{Contactos metal-semiconductor/semiconductor-semiconductor (1 semana)}

\begin{itemize}
	\item Contactos metal-semiconductor: Schottky y Óhmico
	\item La unión PN y electrostática de la juntura
\end{itemize}

\end{frame}

\begin{frame}[t]
\frametitle{Contenidos}

\textbf{El diodo (2.5 semanas)}
	\begin{itemize}
	\item Funcionamiento, curvas características, punto de operación. 
	\item Modelo del diodo ideal, tensión constante, lineal incremental, Shockley.
	\item Línea de carga y punto de operación, resistencia estática y dinámica.
	\item Circuitos de aplicación: rectificadores, reguladores, limitadores. 
	
\end{itemize}

\vspace{3mm}
\textbf{El transistor bipolar BJT (4 semanas)}

\begin{itemize}
	\item Construcción, símbolo y funcionamiento.
	\item Curvas características y polarización.
	\item Modelos del BJT (gran señal, Ebers-Moll y pequeña señal).
	\item Aplicaciones del BJT.
\end{itemize}
\end{frame}

\begin{frame}[t]
\frametitle{Contenidos}

\textbf{El transistor de efecto de campo MOSFET (6 semanas)}

\begin{itemize}
	\item Construcción, símbolo, clasificación del MOSFET.
	\item Funcionamiento, curvas características y polarización.
	\item Efecto de cuerpo, modulación de largo de canal.
	\item Modelo del MOSFET para aplicaciones analógicas.
	\item Aplicaciones analógicas del MOSFET: fuente común, compuerta común, drenador común, ganancia e impedancias de entrada y salida.
	\item Modelo del MOSFET para aplicaciones digitales.
	\item Aplicaciones digitales del MOSFET: compuerta de transmisión, inversor NMOS, inversor CMOS, compuertas lógicas básicas. 
\end{itemize}
\end{frame}

\begin{frame}[t]
\frametitle{Evaluación y Bibliografía}
\textbf{Evaluación} 

\begin{itemize}
	\item Examen parcial I: 30\%
	\item Examen parcial II: 40\%
	\item Examen parcial III: 20\%
	\item Tareas: 10\%
\end{itemize}

\vspace{5mm}
\textbf{Bibliografía obligatoria}

\begin{itemize}
	\item Pedro Julián. Dispositivos Semiconductores Principios y Modelos. Alfaomega, 2013.
	\item Behzad Razavi. Fundamentals of Microelectronics. Wiley, 2a edición, 2013.
	\item Robert F. Pierret. Semiconductor Device Fundamentals. Addison Wesley, 2a edición, 2002.
	\item Jacob Baker. CMOS: Circuit Design, Layout, and Simulation (IEEE Press Series on Microelectronic Systems), Wiley, 4a edición, 2019, ISBN 978-1119481515.
\end{itemize}
\end{frame}


\end{document}

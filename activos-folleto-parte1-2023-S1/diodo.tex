\chapter{El diodo}

\section{Introducción}

\subsection{Construcción, símbolo y funcionamiento}




\section{Modelo analógico}

\subsection{Modelo de CD}

% ToDo: Colocar aquí una intro a los cuatro modelos, e insertar una imagen con las curvas características de cada uno.

\subsubsection{Modelo ideal}

Un diodo ideal conduce si se aplica una tensión $V_D$ positiva, mayor que cero, en polarización directa. Cuando el diodo conduce, se modela como un interruptor cerrado. Si por el contrario se aplica una tensión negativa $V_D$ entre las terminales del diodo, éste se modela como un interruptor abierto.

% ToDo: ejemplo ideal

\subsubsection{Modelo de tensión constante}

En el modelo de tensión constante, se asume que el diodo está encendido si tiene una tensión de al menos $V_{bi}$ entre sus terminales. A partir de esta tensión, el diodo conduce.

% ToDo: ejemplo vconstante

\subsubsection{Modelo lineal incremental}

En este modelo, la corriente que fluye a través del diodo se aproxima por medio de una ecuación lineal:

\[ I_D = g_m (V_D - V_{TH}) \]

Donde $V_{TH}$ es la tensión de umbral necesaria para encender el diodo. Esta tensión es típicamente 0.7 V para diodos de silicio, y 0.4 V para diodos de germanio.

La transconductancia $g_m$ se define como la pendiente de la curva I-V. 

% ToDo: Ejemplo MLI con método iterativo

\subsubsection{Modelo exponencial}

La corriente que fluye a través de un diodo es función de la tensión aplicada:

\[ I_D = I_S \cdot (e^{V_D/V_t} - 1) \]

Donde $I_S$ es la corriente de fuga de reversa, que depende de los parámetros físicos de construcción del dispositivo y se determina con la siguiente ecuación:

\[ I_S = q A n_i^2 \left( \dfrac{D_n}{N_A L_n} + \dfrac{D_p}{N_D L_p} \right) \]

Alternativamente, se puede despejar la tensión del diodo en función de la corriente que fluye a través de la juntura:

\[ V_D = V_t \ln \left( \dfrac{I_C}{I_S} \right) \]

% ToDo: Ejemplo Exp con método iterativo


\subsection{Resistencia estática}

La resistencia estática se calcula como

\[ R_D = \dfrac{V_D}{I_D} \]

\subsection{Modelo de CA}

Punto de operación, transconductancia


\subsection{Resistencia dinámica}

La resistencia dinámica se calcula como

\[ r_d = \dfrac{V_t}{i_d} \]




\section{Aplicaciones del diodo}

\subsection{Rectificador de media onda}

\begin{figure}[H]
    \centering
    \begin{tikzpicture}
        \draw
        (0,2) to[vsourcesin,l=$v_s$] (0,0){}
        (0,2) to[D] (2.5,2){}
        (2,2) to[R,l=$R_L$] (2,0){}
        (0,0) node[ground]{}
        (2,0) node[ground]{}
        (2,2) to[short,-o] (3,2){}
        (3,2) node[anchor=west]{$V_{out}$}
        ;
    \end{tikzpicture}
    \caption{Circuito rectificador de media onda.}   
    \label{rectifier_halfwave} 
\end{figure}

\subsubsection{Condensador de filtrado}

\begin{figure}[H]
    \centering
    \begin{tikzpicture}
        \draw
        (0,2) to[vsourcesin,l=$v_s$] (0,0){}
        (0,2) to[D] (2,2){}
        (2,2) to[C,l=$C_F$] (2,0){}
        (0,0) node[ground]{}
        (2,0) node[ground]{}
        (2,2) to[short,-o] (5,2){}
        (5,2) node[anchor=west]{$V_{out}$}
        (4,2) to[R,l=$R_L$] (4,0){}
        (4,0) node[ground]{}
        ;
    \end{tikzpicture}
    \caption{Circuito rectificador de media onda.}   
    \label{rectifier_halfwave_CF} 
\end{figure}

\subsection{Rectificador de onda completa}

\subsubsection{Condensador de filtrado}

\[ V_r = \dfrac{I_L T}{C_F} \]

\subsection{Rectificador con puente de diodos}

\subsubsection{Condensador de filtrado}

\[ V_r = \dfrac{I_L T}{2 C_F} \]


\subsection{Limitadores de tensión}

\subsubsection{Limitador serie}

\subsubsection{Limitador paralelo}

\begin{figure}[H]
    \centering
    \begin{tikzpicture}
        \draw
        (0,2) to[vsourcesin,l=$v_s$] (0,0){}
        (0,2) to[R,l=,l=$R_S$] (2,2){}
        (2,2) to[D] (2,0){}
        (0,0) node[ground]{}
        (2,0) node[ground]{}
        (2,2) to[short,-o] (3,2){}
        (3,2) node[anchor=west]{$V_{out}$}
        ;
    \end{tikzpicture}
    \caption{Circuito limitador paralelo.}   
    \label{limiter_parallel} 
\end{figure}


\begin{figure}[H]
    \centering
    \begin{tikzpicture}
        \draw
        (0,2) to[vsourcesin,l=$v_s$] (0,-1){}
        (0,2) to[R,l=,l=$R_S$] (2,2){}
        (2,2) to[D] (2,0){}
        (2,0) to[battery2,l=$V_1$] (2,-1){}
        (0,-1) node[ground]{}
        (2,-1) node[ground]{}
        (2,2) to[short,-o] (3,2){}
        (3,2) node[anchor=west]{$V_{out}$}
        ;
    \end{tikzpicture}
    \caption{Circuito limitador paralelo con fuente.}   
    \label{limiter_parallel_V} 
\end{figure}

\begin{figure}[H]
    \centering
    \begin{tikzpicture}
        \draw
        (0,2) to[vsourcesin,l=$v_s$] (0,-1){}
        (0,2) to[R,l=,l=$R_S$] (2,2){}
        (2,2) to[D] (2,0){}
        (2,0) to[battery2,l=$V_1$] (2,-1){}
        (0,-1) node[ground]{}
        (4,0) to[D] (4,2){}
        (4,-1) to[battery2,l_=$V_2$] (4,0){}
        (4,-1) node[ground]{}
        (2,-1) node[ground]{}
        (2,2) to[short,-o] (5,2){}
        (5,2) node[anchor=west]{$V_{out}$}
        ;
    \end{tikzpicture}
    \caption{Circuito limitador paralelo con dos fuentes.}   
    \label{limiter_parallel_2V}
\end{figure}


\subsection{Cambiadores de nivel}

\subsection{Duplicadores de tensión}

\subsection{Compuertas lógicas}

\subsection{Diodo LED}

\subsection{Diodo Zener}

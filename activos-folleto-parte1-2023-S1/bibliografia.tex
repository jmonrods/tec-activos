\begin{thebibliography}{00}
\bibitem{laws2018} Laws, D. (2018). 13 sextillion \& counting: the long \& winding road to the most frequently manufactured human artifact in history. Online: \url{https://computerhistory.org/blog/13-sextillion-counting-the-long-winding-road-to-the-most-frequently-manufactured-human-artifact-in-history/}
\bibitem{moore1965} Moore, Gordon E. (1965). Cramming more components onto integrated circuits, Electronics, 38(8), April 19, 1965, pp. 114, doi=\href{https://dx.doi.org/10.1109/N-SSC.2006.4785860}{10.1109/N-SSC.2006.4785860}.
\bibitem{dennard1974} Dennard, R.H. and Gaensslen, F.H. and Yu, Hwa-Nien and Rideout, V.L. and Bassous, E. and LeBlanc, A.R. (1974). Design of ion-implanted MOSFET's with very small physical dimensions, IEEE Journal of Solid-State Circuits, 9(5), pp. 256-258, doi=\href{https://dx.doi.org/10.1109/JSSC.1974.1050511}{10.1109/JSSC.1974.1050511}.
\bibitem{serway1998} Serway, R. (1998). Principles of physics, 2nd ed., Fort Worth, Saunders College Pub, pp. 602.
\bibitem{b1} B. Razavi (2013). Fundamentals of Microelectronics, 2nd edition, Wiley, ISBN 9781118156322.
%\bibitem{b2} Tsividis, Y. and McAndrew C. (2011). Operation and Modeling of the MOS Transistor, 3rd Ed. Oxford University Press.
%\bibitem{b3} Sze S (2002). M. Semiconductor Devices: Physics and Technology, 2 Ed. Wiley.
\bibitem{b4} Albella (2005). Fundamentos de microelectrónica, nanoelectrónica y fotónica. Prentice Hall.
%\bibitem{b5} Rubio (2000). Diseño de circuitos y sistemas integrados. Alfaomega.
%\bibitem{b6} Sedra, K. Smith (1999). Circuitos Microelectrónicos. Cuarta Edición. Mc. Graw-Hill.
\bibitem{b7} R. Pierret (1996). Semiconductor device fundamentals, 2nd ed., Addison Wesley.
\bibitem{b11} P. Julián (2013). Dispositivos semiconductores: principios y modelos. Alfaomega.
\end{thebibliography}

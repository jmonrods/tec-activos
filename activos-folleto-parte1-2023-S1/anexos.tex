\chapter*{Anexos}

\section*{Constantes físicas y materiales}
\begin{center}
    %\captionof{table}{Constantes}
    \label{tab:Ctes}
    \setlength{\extrarowheight}{9pt}
    \begin{tabular}{ c | c | c | c }
      \hline
      \thead{\large Nombre} &
      \thead{\large Símbolo} & 
      \thead{\large Constante} &
      \thead{\large Unidades}\\
      \hline
      %\hline
      
      \makecell {
      \color{gray}{Boltzmann}\\
      \color{gray}{}\\
      \color{gray}{Planck}\\
      \color{gray}{Carga del electrón}\\
      \color{gray}{Tensión térmica}\\
      \color{gray}{Concentración intrínseca ($Si$)}\\
      \color{gray}{Banda prohibida ($Si$)}\\
      \color{gray}{Permitividad del vacío}\\
      \color{gray}{}\\
      \color{gray}{Permitividad del Silicio ($Si$)}\\
      \color{gray}{Permitividad del Dióxido de Silicio ($SiO_{2}$)}} 
        
      
      & \makecell{
      \color{gray}{$k$}\\
      \color{gray}{}\\
      \color{gray}{$h$}\\
      \color{gray}{$q$}\\
      \color{gray}{$kT/q$}\\
      \color{gray}{$n_{i}$}\\
      \color{gray}{$E_{g}$}\\
      \color{gray}{$\varepsilon _{o}$}\\
      \color{gray}{}\\
      \color{gray}{$\varepsilon _{Si}$}\\
      \color{gray}{$\varepsilon _{ox}$}}  
      
      & \makecell{
      \color{gray}{$1.381\times 10^{-23}$}\\
      \color{gray}{$8.617\times 10^{-5}$}\\
      \color{gray}{$6.626\times 10^{-34}$}\\
      \color{gray}{$1.602\times 10^{-19}$}\\
      \color{gray}{$0.026$}\\
      \color{gray}{$1\times
10^{10}$ ($@ T=300K$)}\\
      \color{gray}{$1.2$ ($@ T=300K$)}\\
      \color{gray}{$8.854\times
10^{-12}$}\\
      \color{gray}{$8.854\times
10^{-14}$}\\
      \color{gray}{$11.7 \varepsilon _{o}$}\\
      \color{gray}{$3.97 \varepsilon _{o}$}
      }
      
      & \makecell{
      \color{gray}{$J\cdot{}K^{-1}$}\\
      \color{gray}{$eV\cdot{}K^{-1}$}\\
      \color{gray}{$J\cdot{}s$}\\
      \color{gray}{C}\\
      \color{gray}{V}\\
      \color{gray}{cm$^{-3}$}\\
      \color{gray}{eV}\\
      \color{gray}{C$^{2}$ N$^{-1}$ m$^{-2}$}\\
      \color{gray}{F cm$^{-1}$}\\
      \color{gray}{F cm$^{-1}$}\\
      \color{gray}{F cm$^{-1}$}
      }
      
      \\
      \hline
      %\hline
    \end{tabular}
\end{center}

\section*{Ecuación de Schrödinger}

La ecuación de onda:

\[ \Psi = e^{i(kx - \omega t)} \]

La primera derivada:

\[ \dfrac{d\Psi}{dx} = ik e^{i(kx - \omega t)} = ik\Psi \]

La segunda derivada:

\[ \dfrac{d^2\Psi}{dx^2} = i^2 k^2 e^{i(kx - \omega t)} = -k^2\Psi \]

La relación de DeBroglie:

\[ k = \dfrac{2\pi}{\lambda} = \dfrac{\rho}{\hbar} \]

Sustituyendo la relación de DeBroglie en la segunda derivada se obtiene:

\[ \dfrac{d^2\Psi}{dx^2} = \dfrac{-\rho^2}{\hbar^2}\Psi \]

Reagrupando términos se obtiene la siguiente expresión:

\[ \boxed{-\hbar^2 \dfrac{d^2\Psi}{dx^2} = \rho^2 \Psi} \]


Ahora escribimos la energía como la suma de energía potencial ($P.E.$) más la energía cinética ($K.E.$):

\begin{eqnarray}
    E = P.E. + K.E. \\
    E = V + \dfrac{1}{2} m v^2 \\
\end{eqnarray}

El momento se define como:

\begin{eqnarray}
    \rho = m v
\end{eqnarray}

Por lo que la energía se escribe como:

\begin{eqnarray}
    E = V + \dfrac{\rho^2}{2m} 
\end{eqnarray}

Ahora se multiplica por la función de onda $\Psi$ a ambos lados:

\begin{eqnarray}
    E\Psi = V\Psi + \dfrac{\rho^2}{2m}\Psi 
\end{eqnarray}

Sustituyendo el resultado que teníamos previamente calculado:

\begin{eqnarray}
    E\Psi = V\Psi + \dfrac{-\hbar}{2m}\dfrac{d^2\Psi}{dx^2} 
\end{eqnarray}

Reacomodando los términos y extendiendo la definición con el operador de gradiente:

\begin{eqnarray}
    \boxed{E\Psi = \dfrac{-\hbar}{2m} \nabla \Psi + V\Psi}
\end{eqnarray}

